\begin{figure}[t]
\centering
\includegraphics[width=3.3in]{so_trend1}
\includegraphics[width=3.3in]{so_trend2}
\caption{Stack Overflow trends of most popular languages
	for last decade(top) and the trends of system-related languages for
	stack overflow(bottom)}
\label{fig:so_trend}
\end{figure}

\section{Motivation}
With the existing dataset, natural language(NL) and programming language(PL)
code snippet pair, most question and answer dataset for low-level PL like C/C++
and
interpreter language related to computer systems like bash shell is not
relatively focused on code generation
and summarization tasks, which is because of meaning representation difficulty.

In prior works about NL to code generation and code summarization
~\cite{xu20aclcodegen, orlanski2021reading, yin2018mining, parvez2021retrieval,
feng2020codebert}. Their approaches are mainly
based on many used PL such as Python, Java, JavaScript, etc. It may be because
those PL have numerical libraries and documentations. Besides, the number of
users using high-level PL is also relatively more significant than users
using low-level PL such as C/C++, Assembly code, and shell
script languages such as sh, csh, and bash.

Figure~\ref{fig:so_trend}(top) shows the most popular PL from Stack Overflow trends for the last
decade using the questions tag. With
the convenience and accessibility, Python has been the top popular language.
After 2014, since Deep Neural Network(DNN)
a framework like TensorFlow and PyTorch are developed and
widely used, Python question ratio of SO significantly increased.

Figure~\ref{fig:so_trend}(bottom) describes the usage of low-level PL and
shell script languages. We can realize questions for those PL
are not frequently asked by users due to a relative lack of
users. But we know that a QA dataset for low-level PL is
needed for many operating systems and computer architecture
researchers and engineers because emerging hardware requires proper software
support.

To expand the dataset for such non-focused PL, we need to
collect from SO QA NL-PL code pair and the various opensource communities for
system software languages
such as Linux Kernel Mailing List(LKML) and Ask Ubuntu
for shell. There is the LKML archive dataset from Keggle~\cite{miasoedov2017lkmlarchive}, but
there are no codes and discussions for this
dataset.

Using the LKML dataset, I anticipate that the sentimental analysis for
committing with review messages from kernel maintainer is possible to exploit
that dataset. another thing
what I proposed method for that dataset is code summarization for code patches.
Most Linux kernel patch has code difference compared
to the previous version of the kernel and commit message describing what they
add new features or remove unnecessary code
for readability or fix kernel panic bugs. We can exploit those
information from kernel patch mail to summarize code or
describing why they optimize code for their goals.
