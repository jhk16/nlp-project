\begin{figure}[t]
\centering
\includegraphics[width=3.3in]{so_trend1}
\includegraphics[width=3.3in]{so_trend2}
\caption{Stack Overflow trends of most popular languages for last
decade(top) and the trends of system-related languages for stack
overflow(bottom)}
\label{fig:so_trend}
\end{figure}

\section{Motivation}
With the exisiting dataset natural language(NL) and programming language(PL) code 
snippet pair, most of question and answer dataset for low level PL like C/C++
and interpreter language related to computer system like bash shell script are
not relatevely focused on code generation and summarization tasks,
which is because of meanning representation difficulty.

In prior works about NL to code generation and code summarization
~\cite{xu20aclcodegen, orlanski2021reading, yin2018mining, parvez2021retrieval,
feng2020codebert}, their approaches are mainly
based on many used PL such as Python, Java, JavaScript, etc. It
may be because those PL have numerical libraries and documentations. Beside, the
number of users using high-level PL are also relatively larger than users using
low-level PL such as C/C++, Assembly code and shell script languages such as sh,
csh and bash.

Figure~\ref{fig:so_trend}(top) show the most popular PL from Stack Overflow trends
for the last decade by using tag of questoin. With the convienience and
accessiblity, Python has been top popular language. After 2014, since Deep
Neural Network(DNN) framework like tensorflow and pytorch are develped and
widly used, Python question ratio of SO significantly increased.

Figure~\ref{fig:so_trend}(bottom) describe the usage of
low-level PL and shell script lanagues. We can realize questions for those PL
are not frequently asked by users due to relatevly lack of users. But we know
that QA dataset for low-level PL is needed to many operating system and computer
architecture researchers and engineers because emerging hardwares require
suitable softwoare supporting.

To expand dataset for such non-focused PL, we need to collect
from not only SO QA NL-PL code pair, but also various open source commnuity for
system software lanagues such as Linux Kernel Mailing List(LKML) and Ask Ubuntu
for shell. There is LKML archive dataset from
Keggle~\cite{miasoedov2017lkmlarchive}, but there are no code and discussions
for this dataset.

By using LKML dataset, I anticipate the
sentimental analysis for commit with review messages from kenrel maintainer is
possible cases to exploit that dataset. another thing what I proposed method for
that dataset is code summarization for code patches.

Most of linux kernel patch
has code difference compared to previous version of kernel and commit message
describing what they add new features or remove unnecessary code for readability
or fix kernel panic bug. We can exploit those information from kernel patch mail
to summarize code or describing why they optimize code for their goals.
